\section{Discusión}
Hemos encontrado que los genes implicados en el fenotipo de diabetes materna interactuaban con el gen GNB3 en mecanismos de regulación de insulina y glucagón. Lo cual está relacionado con problemas de obesidad que pueden afectar a pacientes con este fenotipo. Además, encontramos que un subconjunto de genes del fenotipo estaban implicados en una anormalidad en la visión.

Ya se sospechaba que este gen estaba relacionado con la diabetes materna \cite{Feng2019} ya que ambos están implicados en procesos de regulación de insulina \cite{Rodolaki2023, Feng2019} que pueden producir problemas como la obesidad \cite{Shah2011}. También, un proceso alterado por la diabetes materna es la visión \cite{Nelson1986}, lo que pudimos observar en los resultados. Como se sospechaba hemos encontrado una relación entre la diabetes materna y el gen GNB3 en mecanismo implicados en la obesidad.

Nuestros resultados tiene una buena fuerza estadística ya que todos los procesos que hemos obtenido al hacer el enriquecimiento funcional presentan un p valor y un FDR significativos. También, hemos simplificado la red formada para que solo exista una relación entre los nodos. Esto hace que la búsqueda de comunidades sea mucho más eficiente ya que solo nos interesa si un nodo está relacionado con otro, no si existe más de una evidencia que lo justifica.

Una limitación que presenta el estudio es que como no se obtuvieron procesos relacionados con la insulina en el primer enriquecimiento funcional, se realizó una propagación de la comunidad. Esto es un procedimiento correcto, ya que busca más nodos relacionados con la comunidad de estudio para encontrar más información. Sin embargo, este paso fue en parte premeditado; es decir, la ejecución de la propagación solo se consideró tras obtener y analizar los resultados iniciales.

En vista de nuestros resultados, se abren diversas oportunidades para estudios confirmatorios que consoliden y expandan nuestras observaciones. En primer lugar, recomendamos la validación experimental de las interacciones identificadas entre los genes asociados a la diabetes materna y el gen GNB3 en los mecanismos de regulación de insulina y glucagón. La confirmación de estas interacciones mediante análisis de expresión diferencial\cite{Haynes2013} podría proporcionar una base sólida para futuros estudios clínicos.

Además, es importante abordar las preguntas que aún persisten en nuestra investigación. La propagación de la comunidad, aunque justificada, plantea la interrogante sobre la extensión de la red y la identificación de nuevos nodos relacionados con la diabetes materna. Se requieren estudios adicionales para comprender completamente la complejidad de estas interacciones y su contribución a los fenotipos asociados.

La especulación y recomendación sobre las implicaciones humanas más amplias de nuestros hallazgos son cruciales para contextualizar la relevancia de la investigación. En términos más amplios, la identificación de mecanismos de regulación de insulina y glucagón relacionados con la diabetes materna y la obesidad\cite{Shah2011} mediante el gen GNB3 sugiere posibles objetivos terapéuticos. Esto no solo podría beneficiar a las personas con diabetes materna, sino también a aquellos en riesgo de desarrollar obesidad relacionada con problemas metabólicos. Los resultados de nuestro estudio pueden orientar estrategias preventivas y terapéuticas, mejorando la calidad de vida de las personas afectadas.

