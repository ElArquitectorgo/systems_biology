\section{Discusión}
Hemos encontrado que los genes implicados en la DM interactuaban con el gen GNB3 en mecanismos de regulación de insulina y glucagón, lo cual está relacionado con problemas de obesidad que pueden afectar a pacientes con este fenotipo. Además, en el primer enriquecimiento vimos que existía una relación con problemas de visión, otra característica relacionada con la diabetes \cite{Bailes2002}.

La relación entre el gen GNB3 y la DM se confirma, respaldada por su participación común en los procesos de regulación de insulina \cite{Rodolaki2023, Feng2019}, los cuales han sido previamente asociados con problemas como la obesidad \cite{Shah2011}. La conexión entre la DM y alteraciones en la visión, respaldada por estudios previos \cite{Nelson1986}, se refleja en nuestros resultados.

Hemos obtenido resultados confiables en el enriquecimiento funcional, ya que todos los procesos obtenidos presentan p-valores y FDR significativos.
%Esto
La simplificación de la red, limitándola a relaciones entre nodos únicos, mejora la eficiencia en la identificación de comunidades, focalizando nuestro interés en la existencia de relaciones más que en la cantidad de evidencias justificativas

Una limitación que presenta el estudio es que como no se obtuvieron procesos relacionados con la insulina en el primer enriquecimiento funcional, se realizó una propagación de la comunidad. Esto es un procedimiento correcto, ya que busca más nodos relacionados con la comunidad de estudio para encontrar más información. Sin embargo, este paso fue en parte premeditado; es decir, la ejecución de la propagación sólo se consideró tras obtener y analizar los resultados iniciales.

En vista de nuestros resultados, se abren diversas oportunidades para estudios confirmatorios que consoliden y expandan nuestras observaciones. En primer lugar, recomendamos la validación experimental de las interacciones identificadas entre los genes asociados a la diabetes materna y el gen GNB3 en los mecanismos de regulación de insulina y glucagón. La confirmación de estas interacciones mediante análisis de expresión diferencial \cite{Haynes2013} podría proporcionar una base sólida para futuros estudios clínicos.

Además, es importante abordar las preguntas que aún persisten en nuestra investigación. La propagación de la comunidad, aunque justificada, plantea la pregunta sobre la extensión de la red y la identificación de nuevos nodos relacionados con la diabetes materna. Se requieren estudios adicionales para comprender completamente la complejidad de estas interacciones y su contribución a los fenotipos asociados.

En general, la identificación de mecanismos de regulación de insulina y glucagón relacionados con la DM y la obesidad  \cite{Shah2011} mediante el gen GNB3 sugiere posibles objetivos terapéuticos. Esto no solo podría beneficiar a las personas con DM, sino también a aquellos en riesgo de desarrollar obesidad relacionada con problemas metabólicos. Los resultados de nuestro estudio pueden orientar estrategias preventivas y terapéuticas, mejorando la calidad de vida de las personas afectadas.

