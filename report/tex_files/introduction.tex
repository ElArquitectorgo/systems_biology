\section{Introducción}

La diabetes materna (DM) o diabetes gestacional (HP:0009800), es un trastorno que afecta a la secreción y la función de la insulina, conduciendo a la hiperglucemia \cite{Rodolaki2023}. Se caracteriza por su aparición en mujeres previamente normoglucémicas \cite{Rodolaki2023}, tratándose de cualquier grado de intolerancia a la glucosa que se desarrolle por primera vez durante el embarazo \cite{ADB2009}, y que no sea claramente diabetes manifiesta \cite{Grazia2020}. Durante el embarazo se ve un aumento de hormonas locales y placentarias que conlleva a un estado de resistencia a la insulina, elevando los niveles de glucosa en sangre para soportar las demandas del feto \cite{Plows2018}. Después de un embarazo saludable, la sensibilidad a la insulina vuelve a los niveles previos, mientras que en algunos casos no ocurre así, resultando en DM \cite{Plows2018}.


Se estima que el gasto en salud en personas diabéticas a nivel mundial en 2017 fue de 850 mil millones de dólares \cite{Cho2018} y que las mujeres que padecen diabetes durante la gestación tienen diez veces más riesgo de desarrollar diabetes mellitus tipo 2 (DMT2) que mujeres con un embarazo normal \cite{Vounzoulaki2020} \cite{You2021}. La prevalencia de hiperglucemia en el embarazo entre mujeres de 20 a 49 años es de un 16\% y la cifra va en aumento \cite{Guariguata2014}.

Se asocia a la DM con enfermedades cardiacas en el feto \cite{Depla2021} e incluso con enfermedades cardiovasculares y cerebrovasculares en la madre \cite{Xie2022}.También se ha relacionado con afecciones que actúan como factores de riesgo, como la obesidad\cite{Shah2011} y la DMT2 \cite{Haroush2004}. En ambas, el incremento de citoquinas proinflamatorias es la principal causa de riesgo\cite{Pantham2015}. Otras patologías metabólicas, como el hipotiroidismo\cite{Gong2016} o la hipotiroxinemia materna\cite{Topaloglu2022}, también se han asociado negativamente con la aparición de DM.
Además de diversas complicaciones del recién nacido tras el embarazo\cite{Depla2021}\cite{Metzger2010}, se ha observado cierta predisposición del bebé a desarrollar algún tipo de diabetes neonatal\cite{Dabelea2000}. Esta predisposición, así como la de la madre, aparecen relacionadas con factores genéticos, como los polimorfismos de los genes KCNJ11, KCNQ1\cite{Ao2015} y ABCC8 \cite{Piccini2018} principalmente, también relacionados con la DMT2\cite{Khan2020}.
Por otra parte, se ha visto una potencial relación con el polimorfismo rs5443 
del gen GNB3\cite{Feng2019} implicado en diversos mecanismos relacionados con la ganancia de peso y la obesidad\cite{Hsiao2013}. 

Durante el embarazo, se produce una leve resistencia a la insulina en pos del crecimiento del feto \cite{Kalhan1999}. Esto provoca una adaptación de las células beta del páncreas de la madre para mantener la homeostasis de glucosa, pero si no es suficiente para sobrepasar la resistencia, se produce la DM \cite{Moyce2018}. El polimorfismo rs5443  del gen GNB3 que regula los niveles de glucosa mediante señalización, provoca una subida considerable de peso durante el embarazo y se ha visto altamente relacionado con la DMT2 \cite{Rizvi2016}. Otros mecanismos afectados por la DM son la regulación de niveles de leptina \cite{Perez2020, Xu2014}, la adiponectina \cite{Plows2018, Xu2014}, tejido adiposo \cite{Plows2018, Desoye2021} y la gluconeogénesis \cite{Catalano2014}.

Según lo expuesto pasamos a presentar la siguiente hipótesis en la siguiente sección.

\subsection{Hipótesis}

H1. Existe una asociación estadísticamente significativa entre la DM y el gen GNB3.