\section{Introducción}

La diabetes materna (DM) o diabetes gestacional (HP:0009800), es un trastorno que afecta a la secreción y la función de la insulina, conduciendo a la hiperglucemia \cite{Rodolaki2023}. Se caracteriza por su aparición en mujeres previamente normoglucémicas \cite{Rodolaki2023}, tratándose de cualquier grado de intolerancia a la glucosa que se desarrolle por primera vez durante el embarazo \cite{ADB2009}, y que no sea claramente diabetes manifiesta \cite{Grazia2020}. Durante el embarazo se ve un aumento de hormonas locales y placentarias que conlleva a un estado de resistencia a la insulina, elevando los niveles de glucosa en sangre para soportar las demandas del feto \cite{Plows2018}. Después de un embarazo saludable, la sensibilidad a la insulina vuelve a los niveles previos, mientras que en algunos casos no ocurre así, resultando en DM \cite{Plows2018}.


Se estima que el gasto en salud en personas diabéticas a nivel mundial en 2017 fue de 850 mil millones de dólares \cite{Cho2018} y que las mujeres que padecen diabetes durante la gestación tienen diez veces más riesgo de desarrollar diabetes mellitus tipo 2 (DMT2) que mujeres con un embarazo normal \cite{Vounzoulaki2020} \cite{You2021}. La prevalencia de hiperglucemia en el embarazo entre mujeres de 20 a 49 años es de un 16% y la cifra va en aumento \cite{Guariguata2014}.

Se asocia a la DM con enfermedades cardiacas en el feto \cite{Depla2021} e incluso con enfermedades cardiovasculares y cerebrovasculares en la madre \cite{Xie2022}.
