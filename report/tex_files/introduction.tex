\section{Introducción}

La diabetes materna (DM) o diabetes gestacional (HP:0009800), es un trastorno que afecta a la secreción y la función de la insulina, conduciendo a la hiperglucemia \cite{Rodolaki2023}. Se caracteriza por su aparición en mujeres previamente normoglucémicas \cite{Rodolaki2023}, tratándose de cualquier grado de intolerancia a la glucosa que se desarrolle por primera vez durante el embarazo \cite{ADB2009}, y que no sea claramente diabetes manifiesta \cite{Grazia2020}. Durante el embarazo se ve un aumento de hormonas locales y placentarias que conlleva a un estado de resistencia a la insulina, elevando los niveles de glucosa en sangre para soportar las demandas del feto \cite{Plows2018}. Después de un embarazo saludable, la sensibilidad a la insulina vuelve a los niveles previos, mientras que en algunos casos no ocurre así, resultando en DM \cite{Plows2018}.


Se estima que el gasto en salud en personas diabéticas a nivel mundial en 2017 fue de 850 mil millones de dólares \cite{Cho2018} y que las mujeres que padecen diabetes durante la gestación tienen diez veces más riesgo de desarrollar diabetes mellitus tipo 2 (DMT2) que mujeres con un embarazo normal \cite{Vounzoulaki2020, You2021}. La prevalencia de hiperglucemia en el embarazo entre mujeres de 20 a 49 años es de un 16\% y la cifra va en aumento \cite{Guariguata2014}.

Se asocia a la DM con enfermedades cardiacas en el feto \cite{Depla2021} e incluso con enfermedades cardiovasculares y cerebrovasculares en la madre \cite{Xie2022}. También se ha relacionado con afecciones que actúan como factores de riesgo, como la obesidad \cite{Shah2011} y la DMT2 \cite{Haroush2004}. En ambas, el incremento de citoquinas proinflamatorias es la principal causa de riesgo \cite{Pantham2015}. Otras patologías metabólicas, como el hipotiroidismo \cite{Gong2016} o la hipotiroxinemia materna \cite{Topaloglu2022}, también se han asociado negativamente con la aparición de DM.
Además de diversas complicaciones del recién nacido tras el embarazo \cite{Depla2021, Metzger2010}, se ha observado cierta predisposición del bebé a desarrollar algún tipo de diabetes neonatal \cite{Dabelea2000}. Esta predisposición, así como la de la madre, aparecen relacionadas con factores genéticos, como los polimorfismos de los genes KCNJ11, KCNQ1 \cite{Ao2015} y ABCC8 \cite{Piccini2018} principalmente, también relacionados con la DMT2 \cite{Khan2020}.

Por otra parte, se ha visto una potencial relación con el gen GNB3, el cual codifica la subunidad beta-3 de la proteína G \cite{Feng2019}. Esta proteína desempeña un papel clave en la transducción de señales intracelulares, afectando la respuesta celular a diversos estímulos, como la regulación de la glucosa \cite{Neves2002}. Las alteraciones de esta subunidad se han visto implicadas en diversos mecanismos relacionados con la obesidad \cite{Hsiao2013}, hipertensión y diabetes \cite{Siffert2005}.
 
El gen GNB3, que regula los niveles de glucosa mediante señalización, y se ha asociado directamente a la DMT2 \cite{Rizvi2016}, podría estar altamente implicado en la DM. Se han estudiado polimorsfismos de un solo nucleótido (SNP) que incrementan el riesgo de desarrollar DM \cite{ortega}. Entre ellos, se ha descrito que los genotipos CT y TT  del gen GNB3 podrían estar altamente relacionados con un riesgo alto de DM, y las mujeres con el alelo T tienen una ganancia de peso mayor durante la gestación \cite{ortega}. 


Según lo expuesto pasamos a presentar la siguiente hipótesis en la siguiente sección.

\subsection{Hipótesis}

H1. Existe una asociación estadísticamente significativa entre la DM y el gen GNB3.